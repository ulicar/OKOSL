\documentclass[table,usenames,dvipsnames]{beamer}

\usepackage[english]{babel}
\usepackage[utf8]{inputenc}
\usepackage{listings}
\usepackage{datetime}
\usepackage{graphics}
\usepackage{fancybox}
\usepackage{color}
\usepackage[normalem]{ulem}
\usepackage{tikz}
\usepackage{listings}
\usetikzlibrary{shapes,arrows}
\usetheme{CambridgeUS}
\usecolortheme{seagull}

\DefineNamedColor{named}{Purple}{cmyk}{0.52,0.97,0,0.55}
\setbeamertemplate{itemize item}[triangle]
\setbeamercolor{title}{fg=Purple}
\setbeamercolor{frametitle}{fg=Purple}
\setbeamercolor{itemize item}{fg=Purple}
\setbeamercolor{section number projected}{bg=Purple,fg=white}
\setbeamercolor{subsection number projected}{bg=Purple}

\renewcommand{\dateseparator}{.}
\newcommand{\todayiso}{\twodigit\day \dateseparator \twodigit\month \dateseparator \the\year}
\newcommand{\shell}[1]{\texttt{#1}}
\definecolor{LightGray}{gray}{0.9}

\title{Osnove korištenja operacijskog sustava Linux}
\subtitle{11. Linux u praksi}
\author[Dominik Barbarić]{Dominik Barbarić\\ \small{Nositelj: dr. sc. Stjepan Groš}}
\institute[FER]{Sveučilište u Zagrebu \\
	Fakultet elektrotehnike i računarstva}

\date{\todayiso}

\begin{document}
%\beamerdefaultoverlayspecification{<+->}
{
	\setbeamertemplate{headline}[] % still there but empty
	\setbeamertemplate{footline}{}
	
	\begin{frame}
		\maketitle
	\end{frame}
}

\begin{frame}[t]
	\frametitle{Programi u Linuxu}
	\begin{itemize}
		\item Izvršne datoteke u Linuxu mogu biti u formatu:
		\begin{itemize}
			\item \emph{a.out} - do kernela 1.2
			\item \emph{elf} (Executable and Linkable Format) - od kernela 1.2 nadalje
		\end{itemize}
		\item[]
		\item Uobičajene metode instalacije programa:
		\begin{itemize}
			\item Iz izvornog koda (tarball arhiva i \emph{make} alat)
			\item Pomoću sustava upravljanja paketima
		\end{itemize}
	\end{itemize}
\end{frame}

\begin{frame}[t]
	\frametitle{Programi u Linuxu}
	\framesubtitle{Kompajliranje i instalacija}
	Zašto instaliravati programe iz izvornog koda?
	\begin{itemize}
		\item Linux je multiplatformski OS
		\item Izvorni kod je pisan za određeni OS (ne ovisi o arhitekturi, osim u programima posebne namjene)
		\item Kompajliranjem izvornog koda na odredišnoj arhitekturi dobivamo valjanu izvršnu datoteku
		\item[]
	\end{itemize}
	Distribucija izvornih kodova open-source programa
	\begin{itemize}
		\item Većina izvornih kodova open source programa pisanih za Linux dolaze u tar arhivama
		\item Tar arhive se obično dodatno komprimiraju gzip kompresijskim formatom - \emph{tarball} (.tar.gz / .tgz)
	\end{itemize}
\end{frame}


\begin{frame}[t]
	\frametitle{Programi u Linuxu}
	\framesubtitle{Kompajliranje i instalacija}
	\emph{make}
	\begin{itemize}
		\item Alat koji služi za automatizaciju kompajliranja i instalacije programa
		\item[] Sintaksa: \shell{make <target>}
		\item \shell{make} se oslanja na skriptu \shell{Makefile} koja se nalazi u direktoriju s izvornim kodom
		\item \shell{make} izvršava \emph{target} (ciljni) posao definiran u Makefileu
		\item[]
		\item Za svaki projekt piše se specifični Makefile koji onda izvršava sve potrebne radnje koje omogućuju pokretanje programa (npr. instaliranje posebnih bibilioteka)
		\item Uobičajeni targeti u Makefileu su \emph{all}, \emph{install}, a mogu se definirati i targeti za deinstalaciju programa
	\end{itemize}
\end{frame}

\begin{frame}[t]
	\frametitle{Programi u Linuxu}
	\framesubtitle{Sustav upravljanja paketima}
	\begin{itemize}
		\item Programski paketi se u većini instaliraju \emph{sustavom upravljanja paketima}
		\item Programi dolaze u paketima u kojima se nalaze izvorni kod ili (češće) izvršni program te podaci o koracima koje je potrebno provesti za instalaciju i pokretanje programa - \emph{dependencies}
		\item Sustav upravljanja paketima pamti instalirane pakete te vodi računa o ažuriranju i postupcima za deinstalaciju paketa
	\end{itemize}
\end{frame}

\begin{frame}[t]
	\frametitle{Programi u Linuxu}
	\framesubtitle{Sustavi upravljanja paketima}
	Poznatiji sustavi upravljanja paketima:
	\begin{table}[h]
		\rowcolors{1}{White}{LightGray}
		\begin{tabular}{l l l}
			\rowcolor{BlueViolet!20} Sustav & Distribucije & Format paketa \\
			dpkg & Debian, Ubuntu, Knoppix, ... & .deb \\
			RPM & Red Hat, Fedora, openSUSE, Cent OS, ... & .rpm \\
			pacman & Arch & .tar.xz \\
		\end{tabular}
	\end{table}
	\begin{itemize}
		\item Svaki od ovih sustava se može koristiti pomoću ugrađenih alata ili putem nekog \emph{frontenda}
		\item Primjer - Debian
		\item[] \shell{apt-get install vim}
		\item[] \small Naredba poziva \emph{apt} (Advanced packaging tool - frontend za dpkg) koji pretražuje Debianov repozitorij te preuzima i instalira paket \shell{vim}
	\end{itemize}
\end{frame}

% % dpkg usage

\begin{frame}[t]
	\frametitle{Programi u Linuxu}
	\framesubtitle{Sustavi upravljanja paketima}
	\begin{itemize}
		\item \shell{apt} je frontend za \shell{dpkg}
		\begin{itemize}
			\item Spaja se sa službenim Debian / Ubuntu / \ldots repozitorijem i ostalim repozitorijima navedenim u
			\item[] \shell{/etc/apt/sources.list}
			\item Vodi računa o redovitom ažuriranju instaliranih paketa
			\item Za apt postoje i napredniji frontendi (npr. \emph{Synaptic package manager}, \shell{aptitude}, \ldots)
		\end{itemize}
		\item \shell{dpkg} se može koristiti kao samostalan alat za instalaciju lokalno dostupnih .deb paketa
		\item[]
		\item Zadatak
		\begin{itemize}
			\item Proučiti man stranice za dpkg i isprobati ručnu instalaciju nekog paketa iz službenog repozitorija distribucije
		\end{itemize}
	\end{itemize}
\end{frame}

% DESKTOP

\begin{frame}[t]
	\frametitle{Literatura}
	\begin{itemize}
		\item \url{http://cs.mipt.ru/docs/comp/eng/os/linux/howto/howto\_english/elf/elf-howto-1.html}
		\item \url{http://www.opensourceforu.com/2012/06/gnu-make-in-detail-for-beginners/}
		\item \url{http://www.rpm.org/max-rpm/ch-intro-to-rpm.html}
	\end{itemize}
	
	\begin{itemize}
		\item man a.out
		\item man elf
	\end{itemize}
\end{frame}
	
\end{document}