\documentclass[table,usenames,dvipsnames] {beamer}

\usepackage[english]{babel}
\usepackage[utf8]{inputenc}
\usepackage{listings}
\usepackage{datetime}
\usepackage{graphics}
\usepackage{fancybox}
\usepackage{color}
\usepackage{courier}
\usepackage[normalem]{ulem}
\usepackage{tikz}
\usepackage{verbatim}
\usepackage{multirow}
\usetikzlibrary{shapes,arrows}
\usetheme{CambridgeUS}
\usecolortheme{seagull}
% Changing of bullet foreground color not possible if {itemize item}[ball]
\DefineNamedColor{named}{Purple}{cmyk}{0.52,0.97,0,0.55}
\setbeamertemplate{itemize item}[triangle]
\setbeamercolor{title}{fg=Purple}
\setbeamercolor{frametitle}{fg=Purple}
\setbeamercolor{itemize item}{fg=Purple}
\setbeamercolor{section number projected}{bg=Purple,fg=white}
\setbeamercolor{subsection number projected}{bg=Purple}

\renewcommand{\dateseparator}{.}
\newcommand{\todayiso}{\twodigit\day \dateseparator \twodigit\month \dateseparator \the\year}
\newcommand{\shell}[1]{\texttt{#1}}

\title{Osnove korištenja operacijskog sustava Linux}
\subtitle{07. Korisnici i grupe}
\author[Lucija Petricioli, Josip Žuljević]{Lucija Petricioli, Josip Žuljević\\{\small Nositelj: doc. dr. sc. Stjepan Groš}}
\institute[FER]{Sveučilište u Zagrebu \\
				Fakultet elektrotehnike i računarstva}
				
\date{\todayiso}

\begin{document}
    %\beamerdefaultoverlayspecification{<+->}
{
\setbeamertemplate{headline}[] % still there but empty
\setbeamertemplate{footline}{}

\begin{frame}
\maketitle
\end{frame}
}

\begin{frame}
\frametitle{Sadržaj}
\tableofcontents
\end{frame}

\section{Osnovni pojmovi}
\begin{frame}[t]
\frametitle{Osnovni pojmovi (1)}
\begin{itemize}
  \item Linux je višekorisnički operacijski sustav
  \begin{itemize}
    \item Više korisnika može upotrebljavati jedan sustav
  \end{itemize}
  \item Razlozi zbog kojih jezgra mora razlikovati korisnike
  \begin{itemize}
    \item Zaštita privatnosti
    \item Specifične postavke i podaci
    \item Sprečavanje zlouporabe
    \item Pravedna raspodjela resursa
  \end{itemize}
\end{itemize}
\end{frame}

\begin{frame}[t]
\frametitle{Osnovni pojmovi (2)}
\begin{itemize}
  \item Prijava na sustav
  \begin{itemize}
    \item ``logiranje''
  \end{itemize}
  \item Odmah po prijavi na sustav korisnik je smješten u svoj matični 
        direktorij (HOME directory)
  \begin{itemize}
    \item Korisnik u njemu može pisati/brisati
    \item Garantira se da će biti očuvano kada se korisnik odjavi i 
          ponovno prijavi na sustav
  \end{itemize}
\end{itemize}
\end{frame}

\begin{frame}[t]
\frametitle{Osnovni pojmovi (3)}
\begin{itemize}
  \item Terminal -- U/I naprava za komunikaciju korisnika s računalom
  \item Nekada fizički uređaj, danas programski emulatori
  \begin{itemize}
    \item tty0, tty1, tty2, \ldots
    \begin{itemize}
      \item Označavaju virtualne terminale, dostupni u bilo kojem trenutku
	  \item Između terminala se prebacuje sa Ctrl+Alt+F1\ldots F7
    \end{itemize}
    \item pts/N
    \begin{itemize}
      \item Označavaju pseudoterminale, tj. programske emulacije ili 
                spajanje preko mreže npr. \shell{gnome-terminal}
    \end{itemize}
  \end{itemize}
  \item Izlazak iz terminala moguće je ostvariti sa Ctrl+D ili \shell{logout} naredbom iz terminala
\end{itemize}
\end{frame}


\section{Informacije o korisniku}
\begin{frame}[t]
\frametitle{Naredba \shell{who} (1)}
\begin{itemize}
  \item Naredba može prikazati podatke o korisniku
  \item Primjer ispisa
  \begin{itemize}
    \item[] \shell{\$ who}
    \begin{table}[h]
    \begin{flushleft}
    \shell{
    \begin{tabular}{l l l l l}
      cetko & tty7 & 2010-11-11 & 12:01 & (:0)  \\
      cetko & pts/0 & 2010-11-11 & 17:08 & (:0) \\
      cetko & pts/1 & 2010-11-11 & 17:08 & (:0) \\ 
      cetko & pts/2 & 2010-11-11 & 17:12 & (:0)  
    \end{tabular} }
    \end{flushleft}
    \end{table}
  \end{itemize}
\end{itemize}
\end{frame}

\begin{frame}[t]
\frametitle{Naredba \shell{who} (2)}
\begin{itemize}
  \item Poseban oblik naredbe \shell{who} je \shell{who am i}
  \begin{itemize}
    \item Ispisuje tko je trenutni korisnik na trenutnom terminalu
  \end{itemize}
  \item Varijanta te naredbe je \shell{whoami}
  \begin{itemize}
    \item Ispisuje samo korisničko ime
  \end{itemize}
\end{itemize}
\end{frame}

\begin{frame}[t]
\frametitle{Naredba \shell{finger}}
\begin{itemize}
  \item Drugi način prikaza trenutno aktivnih korisnika
  \item Prikazuje trenutno logirane korisnike, ili prikazuje 
          detaljnije podatke o nekom korisniku
   \item Prikazuje dodatne podatke
   \begin{itemize}
   	\item Čita ih iz datoteka \shell{.project} i \shell{.plan}
   \end{itemize}
   \item Ako joj zadamo parametar pretražuje korisnika
   \begin{itemize}
   	\item Pretraživanje se obavlja po korisničkom imenu i pravom imenu
   \end{itemize}
\end{itemize}
\end{frame}

\begin{frame}[t]
\frametitle{Naredba \shell{w}}
\begin{itemize}
  \item Primjer ispisa
    \begin{table}[h]\footnotesize
    \begin{flushleft}
    \shell{
    \begin{tabular}{l l l l l l l l}
      USER & TTY & FROM & LOGIN@ & IDLE & JCPU & PCPU & WHAT  \\
      cetko & tty7 & :0 & 12:01 & 5:32m & 3:45 & 9.67s & awesome \\
      cetko & pts/0 & :0 & 17:29 & 3:21 & 0.33s & 0.33s & bash \\
      cetko & pts/1 & :0 & 7:31 & 1:06 & 0.33s & 0.33s & bash \\
      cetko & pts/5 & :0 & 17:23 & 0.00s & 0.32s & 0.00s & w
    \end{tabular} }
    \end{flushleft}
    \end{table}
\end{itemize}
\end{frame}

\begin{frame}[t]
\frametitle{Root (1)} 
\begin{itemize}
  \item Jezgra interno radi s brojevima koji su dodijeljeni korisnicima
  \begin{itemize}
    \item Naziva se i UID (engl. \emph{user ID})
    \item 16 ili 32-bitni broj
  \end{itemize}
  \item Jedan korisnik se posebno tretira -\bf{root}
  \begin{itemize}
    \item Administratorski korisnički račun -- UID=0
  \end{itemize}
\end{itemize}
\end{frame}

\begin{frame}[t]
\frametitle{Root (2)} 
\begin{itemize}
  \item Bitno pravilo sigurnosti i opreza
  \begin{itemize}
    \item Nije preporučljivo ulogiravati se i/ili raditi kao root!!!
    \item Raditi kao običan korisnik pa tek kad je nužno prebaciti se na 
          root korisnika
  \end{itemize}
  \item Sve ostale korisnike sustav ne tretira posebno ni na koji način
  \item Root može sve!!
  \item Za privremeno dobivanje administrativnih ovlasti koristi se prefiks \shell {sudo}
\end{itemize}
\end{frame}

\begin{frame}[t]
\frametitle{Mijenjanje korisnika} 
\begin{itemize}
  \item Vrlo bitna naredba \shell{su} (engl. \emph{switch user})
  \item Dva bitna oblika naredbe
  \begin{itemize}
    \item \shell{su <korisnicko ime>}
    \item \shell{su - <korisnicko ime>}
  \end{itemize}
  \item Bez argumenata mijenja korisnika u root
\end{itemize}
\end{frame}

\begin{frame}[t]
\frametitle{Baza passwd (1)} 
\begin{itemize}
  \item Temeljna datoteka s korisnicima je \shell{/etc/passwd}
  \begin{itemize} 
    \item Povezuje korisničko ime i UID
  \end{itemize}
  \item Sadrži i ostale podatke
  \begin{itemize}
    \item Nekada je u njoj bila i lozinka
    \item Vrlo loše sa sigurnosne strane - ne može se zabraniti njeno čitanje jer mnoštvo aplikacija ovisi 
          o podacima u toj datoteci
  \end{itemize}
\end{itemize}
\end{frame}

\begin{frame}[t]
\frametitle{Baza passwd (2)} 
\begin{itemize}
  \item Sadrži jedan zapis po liniji
  \begin{itemize}
    \item Svaka linija sadrži sljedeće informacije razdvojene dvotočkom
    \item Korisničko ime, UID, GID (group ID) , info, matični direktorij,
          ljuska koja se koristi
    \begin{itemize}
      \item[] root:x:0:0:root:/root:/bin/bash
    \end{itemize}
  \end{itemize}
  \item Nije preporučljivo direktno uređivati tu bazu
  \begin{itemize}
    \item Postoji posebna naredba \shell{vipw} koja omogućava uređivanje na
          siguran način
  \end{itemize}
\end{itemize}
\end{frame}
    
\begin{frame}[t]
\frametitle{Baza shadow}
\begin{itemize}
  \item Dodana je \shell{/etc/shadow} datoteka
  \begin{itemize}
    \item Sadrži kriptirane lozinke, te dodatne podatke o njihovom trajanju
    \item Čitljiva je isključivo administratoru/root-u
  \end{itemize}
  \item Korisničko ime, lozinka, niz polja sa podacima o promjeni lozinke
  \begin{itemize}
    \item root:T3RqrzxU1MAH3F3wtuQu/:13284:0:99999:7:::
  \end{itemize}
  \item Veza s passwd datotekom je preko polja s korisničkim imenom
\end{itemize}
\end{frame}

\begin{frame}[t]
\frametitle{Grupe (1)}
\begin{itemize}
  \item Korisnici se grupiraju u korisničke grupe
  \item Radi lakše administracije i dijeljenja podataka
  \item Svaki korisnik ima
  \begin{itemize}
    \item Primarnu grupu
    \begin{itemize}
      \item Zapisana u datoteci \shell{etc/passwd} 
    \end{itemize}
    \item Sekundarne grupe
    \begin{itemize}
      \item  Sve grupe kojima korisnik pripada
      \item  Grupe kod kojih je korisnik naveden u \shell{/etc/group} 
                datoteci
    \end{itemize}
  \end{itemize}
\end{itemize}
\end{frame}

\begin{frame}[t]
\frametitle{Grupe (2)}
\begin{itemize}
  \item Operacijski sustav korisnike i grupe vodi preko brojeva!
  \begin{itemize}
    \item UID (User ID)
    \item GID (Group ID)
  \end{itemize}
  \item Podatke možemo saznati korištenjem naredbe \shell{id}
  \begin{itemize}
	\item uid=1000(user) gid=1000(user) groups=1000(user),4(adm)...
  \end{itemize}
\end{itemize}
\end{frame}

\begin{frame}[t]
\frametitle{Grupe (3)}
\begin{itemize}
  \item Po prijavi na sustav korisnik je u svim grupama
  \begin{itemize}
    \item Prijava u druge grupe s naredbom \shell{newgrp}
  \end{itemize}
  \item Podaci o grupama su u datoteci \shell{/etc/group}
  \begin{itemize}
    \item Svaki redak predstavlja jednu grupu
    \begin{itemize}
      \item Ime grupe, lozinka, GID, lista članova
    \end{itemize}
    \item Lista je popis korisničkih imena odvojenih zarezom
  \end{itemize}
  \item Grupe također imaju posebnu datoteku za lozinke 
        \shell{/etc/gshadow}
\end{itemize}
\end{frame}

\begin{frame}[t]
\frametitle{Manipulacija korisnicima}
\begin{itemize}
  \item Tri osnovne operacije s korisnicima
  \begin{itemize}
    \item Dodavanje novog korisnika
    \begin{itemize}
      \item Naredba \shell{adduser}
    \end{itemize}
    \item Promjena lozinke korisnika
    \begin{itemize}
      \item Naredba \shell{passwd}
    \end{itemize}
    \item Promjena podataka o korisniku
    \begin{itemize}
      \item Naredba \shell{usermod}
    \end{itemize}
    \item Uklanjanje korisnika
    \begin{itemize}
      \item Naredba \shell{deluser}
    \end{itemize}
  \end{itemize}
\end{itemize}
\end{frame}

\section{Stvaranje korisnika}
\begin{frame}[t]
\frametitle{Stvaranje korisnika}
\begin{itemize}
  \item Stvaranje novog korisnika
  \begin{itemize}
    \item[] \shell{\$ adduser <korisnik>}
  \end{itemize}
  \item Dodavanje korisnika postojećoj grupi
  \begin{itemize}
    \item[] \shell{\$ adduser <korisnik> <grupa>}
  \end{itemize}
  \item Stvaranje nove grupe
  \begin{itemize}
    \item[] \shell{\$ adduser --group <grupa>}
    \item[] ili
    \item[] \shell{\$ addgroup <grupa>}
  \end{itemize}
\end{itemize}
\end{frame}

\section{Promjena podataka o korisniku}
\begin{frame}[t]
\frametitle{Promjena podataka o korisniku (1)}
\begin{itemize}
  \item Promjena podataka o korisniku
  \begin{itemize}
    \item Mogu se mjenjati svi podaci
    \begin{itemize}
      \item[] \shell{usermod <opcije> <username>}
    \end{itemize}
    \item Promjena ljuske, opcija \shell{-s <shell>}
    \item Promjena matičnog direktorija, opcija \shell{-d <dir>}
  \end{itemize}
  \item Ljuska korisnika može se promijeniti i  naredbom \shell{chsh}
  \item Naredba \shell{chfn} mijenja podatke o korisnicima poput imena 
        i telefonskog broja
  \begin{itemize}
    \item Ista polja koja se unose kod stvaranja korisnika naredbom 
          \shell{adduser}
  \end{itemize}
\end{itemize}
\end{frame}

\begin{frame}[t]
\frametitle{Promjena podataka o korisniku (2)}
\begin{itemize}
  \item Izmjena lozinke pomoću naredbe \shell{passwd}
  \item Moguće je privremeno onemogućavanje prijave korisnika na sustav
  \begin{itemize}
    \item[] \shell{passwd -l <username>}
    \item Uklanjanje privremenog onemogućavanja korisnika, opcija 
          \shell{-u}
    \item[] \shell{passwd -u <username>} 
  \end{itemize}
\end{itemize}
\end{frame}

\section{Brisanje korisnika}
\begin{frame}[t]
\frametitle{Brisanje korisnika}
\begin{itemize}
  \item Brisanje kreiranog korisnika
  \begin{itemize}
    \item[] \shell{\$ deluser <korisnik>}
  \end{itemize}
  \item Brisanje korisnika iz grupe
  \begin{itemize}
    \item[] \shell{\$ deluser <korisnik> <grupa>}
  \end{itemize}
  \item Brisanje grupe
  \begin{itemize}
    \item[] \shell{\$ deluser --group <grupa>}
    \item[] ili
    \item[] \shell{\$ delgroup}
  \end{itemize}
\end{itemize}
\end{frame}

\begin{frame}[t]
\frametitle{Manipulacija korisnicima (2)}
\begin{itemize}
  \item Kod stvaranja korisnika se može definirati lokacija matičnog 
        direktorija i njegovo brisanje zajedno sa korisnikom
  \item Navedene naredbe su sučelja drugih naredbi
  \begin{itemize}
    \item[] \shell{adduser} $\Rightarrow$ \shell{useradd}
    \item[] \shell{deluser} $\Rightarrow$ \shell{userdel}
    \item[] \shell{addgroup} $\Rightarrow$ \shell{groupadd}
    \item[] \shell{delgroup} $\Rightarrow$ \shell{groupdel}
  \end{itemize}
  \item Sve prethodne akcije se mogu napraviti i navedenim naredbama, na 
        drugačiji način
\end{itemize}
\end{frame}

\section{Bitne datoteke}
\begin{frame}[t]
\frametitle{Bitne datoteke}
\begin{itemize}
  \item Ako kod stvaranja korisnika nisu definirani parametri, koriste se 
        postavke u \shell{/etc/adduser.conf} i \shell{/etc/skel/}
  \begin{itemize}
    \item U matičnom direktoriju se stvaraju predefinirane datoteke
    \begin{itemize}
      \item Izgled datoteka je definiran u direktoriju \shell{/etc/skel}
               (engl. \emph{skeleton})
    \end{itemize}
  \end{itemize}
  \item Zadatak
  \begin{itemize}
    \item Proučiti opcije u datoteci \shell{/etc/adduser.conf}
    \item Izlistati direktorij \shell{/etc/skel} i matični direktorij
  \end{itemize}
\end{itemize}
\end{frame}

\section{Naredbe}
\begin{frame}[t]
	\frametitle{Naredbe}
	\begin{table}[h]
		%  \rowcolors{2}{White}{LightGray}
		\begin{tabular}{|c|l|}
			\hline
			\rowcolor{BlueViolet!20}Naredba & Opis \\
			\hline
			Ctrl+D & odjava iz terminala \\
			\hline
			logout & odjava iz terminala \\
			\hline
			who & prikazuje podatke o korisniku \\
			\hline
			who am i & ispisuje korisnika u trenutnom terminalu \\
			\hline
			whoami & ispisuje isključivo korisničko ime korisnika u terminalu \\
			\hline
			finger & ispisuje trenutno aktivne korisnike \\
			\hline
			su & izmjena korisnika \\
			\hline
			newgrp & prijava u drugu grupu \\
			\hline
			usermod & izmjena podataka o korisniku \\
			\hline
			passwd & promjena korisničke lozinke \\
			\hline
		\end{tabular}
	\end{table}
\end{frame}

\end{document}